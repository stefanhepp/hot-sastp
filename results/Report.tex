\documentclass{article}

\usepackage{algorithm}
\usepackage[noend]{algorithmic}

\usepackage{multirow}
\usepackage{datatool}
\usepackage{graphicx}

\begin{document}
\section{General description of the framework and algorithms}

The solutions are represented as a list of spots to visit. The methods used to visit each spot is also stored in that list, but not the
resting time at the spots. Instead, we keep track of the total amount of stamina needed for the tour and calculate the resting time required 
at each node after the final solution has been selected. 
Since the stamina gained by resting is independent of the spot where the rest is added, this can be done in the following way: 
The total amount of required rest time is 
\[
\textit{total rest time} = \max\left(\frac{\textit{total required stamina} - \textit{initial stamina}}{\textit{habitus}},0\right)
\]
Beginning from the first spot in the tour, the amount of resting time for a spot is calculated as
\[
\textit{rest time} = \min\left(\textit{remaining rest time}, \frac{\textit{max stamina} - \textit{current stamina}}{\textit{habitus}}\right)
\]

By keeping track of the total time required to complete the tour (excluding rest times), the total satisfaction gained after the tour and
the stamina reached at the end of the tour (excluding stamina gained by rests), new solutions can be calculated incrementally. 
To check if a neighborhood step leads to a valid solution, the total tour time is calculated as the tour time excluding rest plus
the total required rest time and compared with the maximum time allowed for the tour.
\medskip

Two greedy heuristics have been implemented that construct an initial solution by inserting spots with methods into the tour that have the highest ratio of gained satisfaction per
used time. The nearest neighbor greedy heuristic adds one of the $k$ nearest spots of the last spot in the tour per step. The insert
heuristic selects one of the $k$ nearest spots of all spots in the tour and inserts it into the tour so that the travel time is minimized.
\medskip

For the local search and the VND, several neighborhoods have been implemented. 

% TODO describe neighborhoods



The local search is configured with one neighborhood. In every iteration it selects a new neighbor from the neighborhood, depending on the
step function. If the new neighbor results in a higher satisfaction, the search continues with that neighbor as the new solution. Otherwise
the search stops for the deterministic \emph{next} and \emph{best} step functions, or continues with the old solution for the \emph{random}
step function. In the latter case, the search terminates if either no improvement has been found for a fixed number of iterations or if a
timeout has been reached.


% TODO very short description of VND, Grasp: neighborhoods used and their order, termination criteria.




%%%%%%%%%%%%%%%%%%%%%%%%%%%%%%%%%%%%%%%%%%%%%%%%%%%%%%%%%%%%%%
%%%%%%% Here we load the files containing the results  %%%%%%%

\DTLloaddb[]{greedyNN}{greedy/AT_GREEDY_NN/greedy_AT_GREEDY_NN.csv}
\DTLloaddb[]{greedyIN}{greedy/AT_GREEDY_IN/greedy_AT_GREEDY_IN.csv}
\DTLloaddb[]{localOne}{localSearch/OneOptRepo.csv }
\DTLloaddb[]{localEdge}{localSearch/EdgeOptReport.csv}
\DTLloaddb[]{localMethod}{localSearch/MethodOptReport.csv}
\DTLloaddb[]{vndDeterministic}{VND/VNDResD.csv}
\DTLloaddb[]{graspOne}{grasp/oneOpt/oneOpt.csv}
\DTLloaddb[]{graspEdge}{grasp/edgeOpt/edgeOpt.csv}
\DTLloaddb[]{graspMethod}{grasp/methodOpt/methodOpt.csv}
\DTLloaddb[]{graspVND}{grasp/vndS.csv}

%%%%%%%%%%%%%%%%%%%%%%%%%%%%%%%%%%%%%%%%%%%%%%%%%%%%%%%%%%%%%%
%%%%%%%%%%% Greedy %%%%%%%%%%%%%%%%%%%%%%%%%%%%%%%%%%%%%%%%%%%

\section{Results}
In the following tables you can find the results of running different configurations on the set of problems. 

% TODO very short description of configurations for each table.
%First of the tables (\ref{tab:greedyNN}), 

\begin{table}[b!]
  \vspace{-6mm}%
  \caption{Computed values using the greedy algorithm, nearest neighbour strategy}
  \label{tab:crit:GreedyNN}
  \setlength{\tabcolsep}{1.5mm}
  \centering
  \begin{tabular}{lrrrrl}
    \bfseries Problem &
    \bfseries Stamina &
    \bfseries Time &
    \bfseries Stam k=10& 
   	\bfseries Time 
    \DTLforeach{greedyNN}{\prob=problem,\stam=stamina,\time=time,\stamin=stamina1,\tim=time1}{%
      \DTLiffirstrow{\\\hline}{\\}%
      \prob & \stam &\time & \stamin & \tim%
    }
    \\\hline
  \end{tabular}
\label{tab:greedyNN}
\end{table}

\begin{table}[b!]
  \vspace{-6mm}%
  \caption{Computed values using the greedy algorithm, insertion neighbour strategy}
  \label{tab:crit:GreedyIN}
  \setlength{\tabcolsep}{1.4mm}
  \centering
  \begin{tabular}{lrrrrl}
    \bfseries Problem &
    \bfseries Stamina &
    \bfseries Time &
    \bfseries Stam k=10& 
   	\bfseries Time 
    \DTLforeach{greedyIN}{\prob=problem,\stam=stamina,\time=time,\stamin=stamina1,\tim=time1}{%
      \DTLiffirstrow{\\\hline}{\\}%
      \prob & \stam &\time & \stamin & \tim%
    }
    \\\hline
  \end{tabular}

\end{table}

%%%%%%%%%%%%%%%%%%%%%%%%%%%%%%%%%%%%%%%%%%%%%%%%%%%%%%%%%%%%%%%%%%%%%%%%%%%%%%%%%%%%%
%%%%%%%%%%%%%%%%%%%%%%%%%%% Local Search %%%%%%%%%%%%%%%%%%%%%%%%%%%%%%%%%%%%%%%%%%%%

\begin{table}[b!]
  \vspace{-6mm}%
  \caption{Computed values using the local search algorithm, using SpotOneOpt}
  \label{tab:crit:localOne}
  \setlength{\tabcolsep}{1.4mm}
  \centering
  \begin{tabular}{lrrrrrrrrrr}
    \bfseries Problem &
    \bfseries St,Next &
    \bfseries Time &
    \bfseries St,Best &
    \bfseries Time &
    \bfseries Rand,k5 & 
    \bfseries dev &
    \bfseries Rand,k10& 
   	\bfseries dev
    \DTLforeach{localOne}{\prob=problem,\next=next,\ti=t1,\best=best,\tii=t2,\ki=k5,\devi=dev5,\kii=k10,\devii=dev10}{%
      \DTLiffirstrow{\\\hline}{\\}%
      \prob & \next &\ti & \best & \tii & \ki & \devi & \kii &\devii%
    }
    \\\hline
  \end{tabular}

\end{table}


\begin{table}[b!]
  \vspace{-6mm}%
  \caption{Computed values using the local search algorithm, using Edge Two Opt}
  \label{tab:crit:localEdge}
  \setlength{\tabcolsep}{1.4mm}
  \centering
  \begin{tabular}{lrrrrrrrrrr}
    \bfseries Problem &
    \bfseries St,Next &
    \bfseries Time &
    \bfseries St,Best &
    \bfseries Time &
    \bfseries Rand,k5 & 
    \bfseries dev &
    \bfseries Rand,k10& 
   	\bfseries dev
    \DTLforeach{localEdge}{\prob=problem,\next=next,\ti=t1,\best=best,\tii=t2,\ki=k5,\devi=dev5,\kii=k10,\devii=dev10}{%
      \DTLiffirstrow{\\\hline}{\\}%
      \prob & \next &\ti & \best & \tii & \ki & \devi & \kii &\devii%
    }
    \\\hline
  \end{tabular}

\end{table}

\begin{table}[b!]
  \vspace{-6mm}%
  \caption{Computed values using the local search algorithm, using Method Two Opt}
  \label{tab:crit:localMethod}
  \setlength{\tabcolsep}{1.4mm}
  \centering
  \begin{tabular}{lrrrrrrrr}
    \bfseries Problem &
    \bfseries St,Next &
    \bfseries Time &
    \bfseries Rand,k5 & 
    \bfseries dev &
    \bfseries Rand,k10& 
   	\bfseries dev
    \DTLforeach{localMethod}{\prob=problem,\next=next,\ti=t1,\ki=k5,\devi=dev5,\kii=k10,\devii=dev10}{%
      \DTLiffirstrow{\\\hline}{\\}%
      \prob & \next &\ti & \ki & \devi & \kii &\devii%
    }
    \\\hline
  \end{tabular}

\end{table}
%%%%%%%%%%%%%%%%%%%%%%%%%%%%%%%%%%%%%%%%%%%%%%%%%%%%%%%%%%%%%%%%%%%%%%%%%%%%%%%%%%%%%
%%%%%%%%%%%%%%%%%%%%% VND %%%%%%%%%%%%%%%%%%%%%%%%%%%%%%%%%%%%%%%%%%%%%%%%%%%%%%%%%%%

\begin{table}[b!]
  \vspace{-6mm}%
  \caption{Computed values using the VND, using step function Next and Best }
  \label{tab:crit:vndDeterministic}
  \setlength{\tabcolsep}{1.4mm}
  \centering
  \begin{tabular}{lrrrrrrrrrr}
    \bfseries Problem &
    \bfseries Next &
    \bfseries K=5 &
    \bfseries K=15 &
    \bfseries AvgTime &
    \bfseries Best & 
    \bfseries K=5 &
    \bfseries K=15& 
   	\bfseries AvgTime
    \DTLforeach{vndDeterministic}{\prob=problem,\next=next,\nkv=nk5,\nkxv=nk15,\avgti=avgt1,\best=best,\kv=k5,\kxv=k15,\avgtii=avgt2}{%
      \DTLiffirstrow{\\\hline}{\\}%
      \prob & \next &\nkv & \nkxv & \avgti & \best & \kv & \kxv &\avgtii%
    }
    \\\hline
  \end{tabular}

\end{table}

%%%%%%%%%%%%%%%%%%%%%%%%%%%%%%%%%%%%%%%%%%%%%%%%%%%%%%%%%%%%%%%%%%%%%%%%%%%%%%%%%%%%%
%%%%%%%%%%%%%%%%% GRASP %%%%%%%%%%%%%%%%%%%%%%%%%%%%%%%%%%%%%%%%%%%%%%%%%%%%%%%%%%%%%

\begin{table}[b!]
  \vspace{-6mm}%
  \caption{Grasp using the local search, oneOpt, using step function Next and Best and Random }
  \label{tab:crit:vndDeterministic}
  \setlength{\tabcolsep}{1.4mm}
  \centering
  \begin{tabular}{lrrrrrrrrrr}
    \bfseries Problem &
    \bfseries Best &
    \bfseries Dev &
    \bfseries Next &
    \bfseries Dev &
    \bfseries Random & 
    \bfseries Dev
    \DTLforeach{graspOne}{\prob=problem,\best=best,\devb=devb,\next=next,\devn=devn,\rand=rand,\devr=devr}{%
      \DTLiffirstrow{\\\hline}{\\}%
      \prob &\best & \devb & \next & \devn & \rand & \devr%
    }
    \\\hline
  \end{tabular}

\end{table}

\begin{table}[b!]
  \vspace{-6mm}%
  \caption{Grasp using the local search, edgeOption, using step function Next and Random }
  \label{tab:crit:vndDeterministic}
  \setlength{\tabcolsep}{1.4mm}
  \centering
  \begin{tabular}{lrrrrrrrrrr}
    \bfseries Problem &
    \bfseries Best &
    \bfseries Dev &
    \bfseries Next &
    \bfseries Dev &
    \bfseries Random & 
    \bfseries Dev
    \DTLforeach{graspEdge}{\prob=problem,\best=best,\devb=devb,\next=next,\devn=devn,\rand=rand,\devr=devr}{%
      \DTLiffirstrow{\\\hline}{\\}%
      \prob &\best & \devb & \next & \devn & \rand & \devr%
    }
    \\\hline
  \end{tabular}

\end{table}

\begin{table}[b!]
  \vspace{-6mm}%
  \caption{Grasp using the local search , methodOpt and step function Next and Best }
  \label{tab:crit:vndDeterministic}
  \setlength{\tabcolsep}{1.4mm}
  \centering
  \begin{tabular}{lrrrrrrrrrr}
    \bfseries Problem &
    \bfseries Next &
    \bfseries Dev &
    \bfseries Random &
    \bfseries Dev  
    \DTLforeach{graspMethod}{\prob=problem,\next=next,\dev=dev,\random=random,\devr=devr}{%
      \DTLiffirstrow{\\\hline}{\\}%
      \prob & \next &\dev & \random & \devr %
    }
    \\\hline
  \end{tabular}

\end{table}

\begin{table}[b!]
  \vspace{-6mm}%
  \caption{Grasp using the VND with step function Next and Best }
  \label{tab:crit:vndDeterministic}
  \setlength{\tabcolsep}{1.4mm}
  \centering
  \begin{tabular}{lrrrrrrrrrr}
    \bfseries Problem &
    \bfseries Next &
    \bfseries Dev &
    \bfseries AvgTime &
    \bfseries Best & 
    \bfseries Dev &
    \bfseries AvgTime
    \DTLforeach{graspVND}{\prob=problem,\next=next,\dev=dev,\time=time,\best=best,\bdev=bdev,\btime=btime}{%
      \DTLiffirstrow{\\\hline}{\\}%
      \prob & \next &\dev & \time & \best & \bdev & \btime%
    }
    \\\hline
  \end{tabular}

\end{table}

% We could just put the description of the tables and the discussion into same section, i.e. just refer to the tables in the discusssion, no
% need to describe them first separately?
\section{Discussion}

% TODO compare various configurations
% - single neighborhoods not good enough if tour length does not change
% - Order of neighborhoods in VND important for runtime, maybe for quality of results (TODO check results..)
% - describe best configuration



\end{document}
