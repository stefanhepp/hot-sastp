\documentclass{article}

\usepackage{algorithm}
\usepackage[noend]{algorithmic}

\usepackage{multirow}
\usepackage{datatool}
\usepackage{graphicx}

\usepackage[margin=1in]{geometry}


% TODO Add title
% VU Heuristic Optimization
%  SASTP, Exercise 2
%  Ioan Dragan, 
%  Stefan Hepp, 0026640

\title{SASTP Exercise 2\\
{VU Heuristic Optimization Techniques} }
\author{Stefan Hepp, 0026640 \\
Ioan Dragan, 0856561 }

\begin{document}
\maketitle
\section{General description of the framework and algorithms}

\paragraph{Pheromone matrix}

The pheromone matrix uses pairs consisting of a spot-index and a method-index as indices. It is therefore represented as a four-dimensional
matrix. In order to reduce the memory requirements, the matrix has been implemented as a sparse tree containing rows and columns of the
matrix as nodes. The leaves contain the values $\tau_{ij}$. Nodes are only created when an entry of a leaf is set to some value. For all
$\tau_{ij}$ that do not yet have an entry in the matrix, a default value $\tau_d$ is returned. $\tau_d$ is set to the initial $\tau_0$ value 
and updated during evaporation. The semantics of the entries in the pheromone matrix depends on the ant implementation that is used.

The pheromone matrix is updated by all the ants in sequential order only after the evaporation step, i.e., the matrix is not updated
concurrently.

\paragraph{Ant colony optimization}

\paragraph{Ants}

Two types of ant are implemented in the current version of our solution. One of them constructs a tour in a greedy nearest neighbor way. This means that we restrict the list of spots an ant can visit at a specific time. This limitation is given by the k-nearest nodes. Selection of the next node to be inserted in the tour is done taking into consideration the pheromone matrix value and all the other parameters which are needed in the decision making. The insertion of the new node in the tour is done always at the end. In order to update the pheromone matrix at the end, we only need to traverse the tour once and simply modify the pheromone for each of the vertices taken in the tour.

The other type of ant is based on the best insertion heuristic. This heuristic was one of the heuristics which proved to give good results if used in the classical search algorithms. The heuristic tries to fit the node in the tour such that we maximize the satisfaction. The insertion place is based on the shortest distance from any node in the tour to the node to be inserted.Both implementations of ants do not allow us to end up with unfeasible tours. This is taken care of in the implementation of the ant itself. 


%%%%%%%%%%%%%%%%%%%%%%%%%%%%%%%%%%%%%%%%%%%%%%%%%%%%%%%%%%%%%%
%%%%%%%%%%% Greedy %%%%%%%%%%%%%%%%%%%%%%%%%%%%%%%%%%%%%%%%%%%

\section{Results and discussion}


\begin{table}[b!]
  \vspace{-6mm}%
  \caption{Computed values using the nearest tour neighborhood}
  \label{tab:NearestTour}
  \setlength{\tabcolsep}{1.4mm}
  \centering
  \begin{tabular}{lrrrrrr}
   \multirow{2}{*}{\bfseries Problem} &
      \multicolumn{4}{c}{\bfseries -a 6 -H 0 -I -conf1 } \\
    &
    \bfseries Max. Satis. &
    \bfseries Avg Satis. &
    \bfseries Std. dev. &
    \bfseries Avg. Time 
    \\\hline
    sastp10 & 41.8889 & 41.2037 & 1.6444 & 0.03533 \\ 
	sastp20 & 71.2446 & 61.2053 & 4.6175 & 0.0286 \\ 
	sastp50 & 187.3030 & 179.0910 & 3.8314 & 0.0589 \\ 
	sastp100 & 446.0060 & 423.6268 & 7.9482 & 0.1543 \\ 
	sastp200 & 806.0900 & 781.5946 & 12.5164 & 0.2786 \\ 
	sastp500 & 2146.1300 & 2091.1350 & 43.9897 & 0.8942 \\ 
	sastp1000 & 4213.3100 & 4158.8880 & 30.1404 & 0.9029 

    \\\hline
  \end{tabular}

\end{table}


\begin{table}[b!]
  \vspace{-6mm}%
  \caption{Computed values using the nearest tour neighborhood}
  \label{tab:NearestTour}
  \setlength{\tabcolsep}{1.4mm}
  \centering
  \begin{tabular}{lrrrrrr}
   \multirow{2}{*}{\bfseries Problem} &
      \multicolumn{4}{c}{\bfseries -a 8 -H 0 -I  conf2} \\
    &
    \bfseries Max. Satis. &
    \bfseries Avg Satis. &
    \bfseries Std. dev. &
    \bfseries Avg. Time 
    \\\hline
   sastp10 & 43.2979 & 43.2979 & 0.0 & 0.0364 \\ 
sastp20 & 72.4877 & 72.4877 & 0.0 & 0.0893 \\ 
sastp50 & 212.625 & 211.7637 & 0.7064 & 0.7037 \\ 
sastp100 & 505.092 & 503.7117 & 0.727 & 4.1661 \\ 
sastp200 & 975.133 & 972.3051 & 1.5835 & 22.8307 \\ 
sastp500 & 2617.15 & 2615.308 & 1.0154 & 268.6628 \\ 
sastp1000 & 5451.31 & 5450.006 & 0.9722 & 1023.88
    \\\hline
  \end{tabular}

\end{table}


\begin{table}[b!]
  \vspace{-6mm}%
  \caption{Computed values using the nearest tour neighborhood}
  \label{tab:NearestTour}
  \setlength{\tabcolsep}{1.4mm}
  \centering
  \begin{tabular}{lrrrrrr}
   \multirow{2}{*}{\bfseries Problem} &
      \multicolumn{4}{c}{\bfseries -a 6 -H 0 - I -P 0.2 -conf3 } \\
    &
    \bfseries Max. Satis. &
    \bfseries Avg Satis. &
    \bfseries Std. dev. &
    \bfseries Avg. Time 
    \\\hline
    sastp10 & 41.8889 & 37.8786 & 3.6177 & 0.0371 \\ 
sastp20 & 65.1206 & 50.4986 & 5.0134 & 0.0399 \\ 
sastp50 & 175.668 & 166.6757 & 4.3926 & 0.0544 \\ 
sastp100 & 426.036 & 392.8171 & 11.9743 & 0.0809 \\ 
sastp200 & 752.137 & 733.3267 & 9.7063 & 0.1364 \\ 
sastp500 & 2012.81 & 1987.088 & 19.2344 & 0.4782 \\ 
sastp1000 & 4256.67 & 4160.224 & 63.6609 & 0.8518

    \\\hline
  \end{tabular}

\end{table}


\begin{table}[b!]
  \vspace{-6mm}%
  \caption{Computed values using the nearest tour neighborhood}
  \label{tab:NearestTour}
  \setlength{\tabcolsep}{1.4mm}
  \centering
  \begin{tabular}{lrrrrrr}
   \multirow{2}{*}{\bfseries Problem} &
      \multicolumn{4}{c}{\bfseries -a 8 -H 0 - I -P 0.2 -conf4 } \\
    &
    \bfseries Max. Satis. &
    \bfseries Avg Satis. &
    \bfseries Std. dev. &
    \bfseries Avg. Time 
    \\\hline
sastp10 & 43.2979 & 43.2979 & 0.0 & 0.0436 \\ 
sastp20 & 72.4877 & 72.4374 & 0.2709 & 0.0705 \\ 
sastp50 & 212.625 & 211.0882 & 0.6558 & 0.4623 \\ 
sastp100 & 504.831 & 504.1002 & 0.8066 & 2.3433 \\ 
sastp200 & 977.117 & 973.6227 & 1.3481 & 20.8448 \\ 
sastp500 & 2620.01 & 2616.652 & 1.7338 & 158.866 \\ 
sastp1000 & 5455.27 & 5453.776 & 0.8323 & 1009.884

    \\\hline
  \end{tabular}

\end{table}


\begin{table}[b!]
  \vspace{-6mm}%
  \caption{Computed values using the nearest tour neighborhood}
  \label{tab:NearestTour}
  \setlength{\tabcolsep}{1.4mm}
  \centering
  \begin{tabular}{lrrrrrr}
   \multirow{2}{*}{\bfseries Problem} &
      \multicolumn{4}{c}{\bfseries -a 6 -H 0 -W 1 -U 5 -D -11 -conf5 } \\
    &
    \bfseries Max. Satis. &
    \bfseries Avg Satis. &
    \bfseries Std. dev. &
    \bfseries Avg. Time 
    \\\hline
   sastp10 & 41.8889 & 38.3187 & 2.9868 & 0.0384 \\ 
sastp20 & 65.3593 & 55.1009 & 5.4497 & 0.0448 \\ 
sastp50 & 179.799 & 169.2874 & 6.9689 & 0.0776 \\ 
sastp100 & 426.222 & 406.5479 & 13.8004 & 0.1453 \\ 
sastp200 & 795.61 & 762.5328 & 17.1739 & 0.3086 \\ 
sastp500 & 2062.45 & 2019.754 & 32.3927 & 0.7834 \\ 
sastp1000 & 4153.76 & 4132.17 & 22.0222 & 1.3032
    \\\hline
  \end{tabular}

\end{table}


\begin{table}[b!]
  \vspace{-6mm}%
  \caption{Computed values using the nearest tour neighborhood}
  \label{tab:NearestTour}
  \setlength{\tabcolsep}{1.4mm}
  \centering
  \begin{tabular}{lrrrrrr}
   \multirow{2}{*}{\bfseries Problem} &
      \multicolumn{4}{c}{\bfseries -a 8 -H 0 -W 1 -U 5 -D -11 -conf6 } \\
    &
    \bfseries Max. Satis. &
    \bfseries Avg Satis. &
    \bfseries Std. dev. &
    \bfseries Avg. Time 
    \\\hline
sastp10 & 43.2979 & 43.2979 & 0.0 & 0.0478 \\ 
sastp20 & 72.4877 & 72.4877 & 0.0 & 0.0965 \\ 
sastp50 & 212.625 & 211.6224 & 0.6754 & 0.7246 \\ 
sastp100 & 504.738 & 503.3902 & 0.8192 & 5.0302 \\ 
sastp200 & 974.28 & 970.9718 & 1.7374 & 22.1149 \\ 
sastp500 & 2616.44 & 2615.06 & 1.2754 & 291.8544 \\ 
sastp1000 & 5449.18 & 5447.85 & 1.0035 & 1025.282
    \\\hline
  \end{tabular}

\end{table}

\begin{table}[b!]
  \vspace{-6mm}%
  \caption{Computed values using the nearest tour neighborhood}
  \label{tab:NearestTour}
  \setlength{\tabcolsep}{1.4mm}
  \centering
  \begin{tabular}{lrrrrrr}
   \multirow{2}{*}{\bfseries Problem} &
      \multicolumn{4}{c}{\bfseries -a 6 -H 0 -W 1 -U 5 -D -11 -P 0.2 -conf7 } \\
    &
    \bfseries Max. Satis. &
    \bfseries Avg Satis. &
    \bfseries Std. dev. &
    \bfseries Avg. Time 
    \\\hline
   sastp10 & 41.8889 & 34.8167 & 2.6143 & 0.0369 \\ 
sastp20 & 59.0034 & 49.2742 & 5.3977 & 0.0398 \\ 
sastp50 & 171.452 & 159.399 & 5.1703 & 0.055 \\ 
sastp100 & 401.588 & 375.0972 & 11.5001 & 0.0789 \\ 
sastp200 & 749.305 & 718.628 & 13.3477 & 0.1311 \\ 
sastp500 & 1971.59 & 1941.262 & 17.1473 & 0.4418 \\ 
sastp1000 & 4131.34 & 4097.774 & 25.8643 & 0.796

    \\\hline
  \end{tabular}

\end{table}

\begin{table}[b!]
  \vspace{-6mm}%
  \caption{Computed values using the nearest tour neighborhood}
  \label{tab:NearestTour}
  \setlength{\tabcolsep}{1.4mm}
  \centering
  \begin{tabular}{lrrrrrr}
   \multirow{2}{*}{\bfseries Problem} &
      \multicolumn{4}{c}{\bfseries -a 8 -H 0 -W 1 -U 5 -D -11 -P 0.2 -conf8 } \\
    &
    \bfseries Max. Satis. &
    \bfseries Avg Satis. &
    \bfseries Std. dev. &
    \bfseries Avg. Time 
    \\\hline
  sastp10 & 43.2979 & 43.2979 & 0.0 & 0.0437 \\ 
sastp20 & 72.4877 & 72.0374 & 0.8205 & 0.0759 \\ 
sastp50 & 212.208 & 210.8661 & 0.6092 & 0.4394 \\ 
sastp100 & 504.738 & 503.4019 & 1.034 & 1.6397 \\ 
sastp200 & 977.164 & 973.3107 & 2.1298 & 11.3303 \\ 
sastp500 & 2626.18 & 2623.59 & 2.2716 & 194.2388 \\ 
sastp1000 & 5460.0 & 5455.764 & 2.6621 & 1011.9152
    \\\hline
  \end{tabular}

\end{table}

\begin{table}[b!]
  \vspace{-6mm}%
  \caption{Computed values using the nearest tour neighborhood}
  \label{tab:NearestTour}
  \setlength{\tabcolsep}{1.4mm}
  \centering
  \begin{tabular}{lrrrrrr}
   \multirow{2}{*}{\bfseries Problem} &
      \multicolumn{4}{c}{\bfseries -a 6 -H 1 -I -conf9 } \\
    &
    \bfseries Max. Satis. &
    \bfseries Avg Satis. &
    \bfseries Std. dev. &
    \bfseries Avg. Time 
    \\\hline
sastp10 & 41.8889 & 40.8904 & 2.1519 & 0.0401 \\ 
sastp20 & 70.8606 & 60.4437 & 5.1397 & 0.0808 \\ 
sastp50 & 183.145 & 173.9262 & 5.1244 & 0.2088 \\ 
sastp100 & 415.008 & 397.928 & 9.5908 & 0.7563 \\ 
sastp200 & 749.021 & 733.0668 & 7.7951 & 1.9684 \\ 
sastp500 & 1943.38 & 1934.38 & 8.0681 & 16.9356
    \\\hline
  \end{tabular}

\end{table}

\begin{table}[b!]
  \vspace{-6mm}%
  \caption{Computed values using the nearest tour neighborhood}
  \label{tab:NearestTour}
  \setlength{\tabcolsep}{1.4mm}
  \centering
  \begin{tabular}{lrrrrrr}
   \multirow{2}{*}{\bfseries Problem} &
      \multicolumn{4}{c}{\bfseries -a 6 -H 1 -I -P 0.2 -conf10 } \\
    &
    \bfseries Max. Satis. &
    \bfseries Avg Satis. &
    \bfseries Std. dev. &
    \bfseries Avg. Time 
    \\\hline
sastp10 & 41.8889 & 36.5548 & 3.029 & 0.0415 \\ 
sastp20 & 61.5421 & 51.7632 & 4.9557 & 0.0476 \\ 
sastp50 & 168.5 & 160.2911 & 4.5742 & 0.1381 \\ 
sastp100 & 403.165 & 382.9565 & 9.0094 & 0.5625 \\ 
sastp200 & 746.764 & 726.4918 & 8.7285 & 1.9258 \\ 
sastp500 & 1914.76 & 1906.106 & 7.7125 & 15.2153 \\ 
sastp1000 & 4054.54 & 4026.512 & 18.0058 & 115.5717
    \\\hline
  \end{tabular}

\end{table}

%%%%%%%%%%%%%%%%%%%%%%%%%%%





\begin{figure}[htb]
\centering
\includegraphics{graphs/greedy.pdf}
\caption{Greedy algorithm}
\label{fig:greedy}
\end{figure}

\begin{figure}[htb]
\centering
\includegraphics{graphs/local.pdf}
\caption{Local Search with different neighborhood configurations}
\label{fig:localSearch}
\end{figure}

\begin{figure}[htb]
\centering
\includegraphics{graphs/vnd.pdf}
\caption{VND with different step functions}
\label{fig:vnd}
\end{figure}

\begin{figure}[htb]
\centering
\includegraphics{graphs/grasp.pdf}
\caption{Grasp in different configurations}
\label{fig:grasp}
\end{figure}



\end{document}
